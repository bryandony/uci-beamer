
% Copyright 2004 by Till Tantau <tantau@users.sourceforge.net>.
%
% In principle, this file can be redistributed and/or modified under
% the terms of the GNU Public License, version 2.
%
% However, this file is supposed to be a template to be modified
% for your own needs. For this reason, if you use this file as a
% template and not specifically distribute it as part of a another
% package/program, I grant the extra permission to freely copy and
% modify this file as you see fit and even to delete this copyright
% notice.

\documentclass{beamer}
\usepackage{verbatim}
\usepackage{ulem}
\usepackage{textpos}
\usepackage{lmodern}


% This theme is based on Berkeley beamer theme
% Hide the left bar
\usetheme[width=0in]{Berkeley}
\makeatletter
\beamer@headheight=1.5\baselineskip
\makeatother

\definecolor{myblue}{rgb}{0.16,0.23,0.75}
\definecolor{spartanred}{RGB}{166,25,46}
\definecolor{spartangold}{RGB}{197, 183, 131}
\setbeamercolor{linecolor}{fg=black,bg=spartanred}



% Set the page footline
\makeatletter
\setbeamertemplate{footline}
{
  \leavevmode%
  \hbox{\fontsize{9}{9}\selectfont%
  \begin{beamercolorbox}[wd=.34\paperwidth,ht=2.25ex,dp=1ex,center]{linecolor}
    \usebeamerfont{author in head/foot}\insertshortauthor~~\beamer@ifempty{\insertshortinstitute}{}{(\insertshortinstitute)}
  \end{beamercolorbox}%
  \begin{beamercolorbox}[wd=.33\paperwidth,ht=2.25ex,dp=1ex,center]{linecolor}\usebeamerfont{title in head/foot}\insertshorttitle
  \end{beamercolorbox}%
  \begin{beamercolorbox}[wd=.33\paperwidth,ht=2.25ex,dp=1ex,right]{linecolor}
    \usebeamerfont{date in head/foot}\insertshortdate{}\hspace*{1em}% original: 2ex
    \insertframenumber{} / \inserttotalframenumber\hspace*{1ex}% original: 2ex
  \end{beamercolorbox}}%
  \vskip0pt%
}
\makeatother

% Get rid of the bottom navigation
\beamertemplatenavigationsymbolsempty

% Set the colors
% Title Frame
\setbeamercolor{title}{fg=white,bg=spartanred}
% Blocks
\setbeamercolor{frametitle}{fg=white,bg=spartanred}
\setbeamercolor{block title}{fg=white,bg=black!50}
\setbeamercolor{block title example}{fg=white,bg=spartangold}
\setbeamercolor{block title alerted}{fg=white,bg=spartanred}

% Flatbox added by Sajjad
\definecolor{myred}{rgb}{0.75,0,0}
\setbeamercolor{err_block_color}{fg=white,bg=spartanred}
\setbeamercolor{normal_block_color}{fg=white,bg=black!60}
\setbeamercolor{ex_block_color}{fg=black,bg=spartangold}
\newenvironment{flatbox}[1]{\begin{beamercolorbox}[colsep*=.75ex,vmode]{#1}}{\end{beamercolorbox}}

\setbeamercolor*{structure}{bg=white,fg=black}

\title[SDSU Beamer]{A Beamer Template for Aztecs!}
\author[Monte]{Montezuma}
\institute[SDSU]{Department of Computer Science, San Diego State University}

% A subtitle is optional and this may be deleted
\subtitle{}

% - Use the \inst command only if there are several affiliations.
% - Keep it simple, no one is interested in your street address.

\date{October 2020}
% - Either use conference name or its abbreviation.
% - Not really informative to the audience, more for people (including
%   yourself) who are reading the slides online

\subject{Computer Science}
% This is only inserted into the PDF information catalog. Can be left
% out.

% If you have a file called "university-logo-filename.xxx", where xxx
% is a graphic format that can be processed by latex or pdflatex,
% resp., then you can add a logo as follows:

% \pgfdeclareimage[height=0.5cm]{university-logo}{university-logo-filename}
% \logo{\pgfuseimage{university-logo}}

% Delete this, if you do not want the table of contents to pop up at
% the beginning of each subsection:
\AtBeginSubsection[]
{
  \begin{frame}<beamer>{Outline}
    \tableofcontents[currentsection,currentsubsection]
  \end{frame}
}

% Let's get started

\begin{document}

{
% Remoe the headline on the title page
\setbeamertemplate{headline}{}

\begin{frame}
  \titlepage
\end{frame}
}


% Put UCI logo on top of pages
\addtobeamertemplate{frametitle}{}{%
\begin{textblock*}{100mm}(.83\textwidth,-0.8cm)
\includegraphics[height=0.6cm]{sdsu.png}
\end{textblock*}}

\begin{frame}{Outline}
  \tableofcontents
  % You might wish to add the option [pausesections]
\end{frame}


% Section and subsections will appear in the presentation overview
% and table of contents.
\section{Section 1}


\begin{frame} {Frame 1}
\begin{itemize}
\item item 1
\item item 2
	\begin{enumerate}
		\item subitem
	\end{enumerate}
\end{itemize}
\begin{block}{A block}
This is a normal block.
\end{block}
\begin{example}
This is how example blocks look.
\end{example}
\begin{alertblock}{Another block}
This is how alerted blocks look.
\end{alertblock}
\end{frame}



% All of the following is optional and typically not needed.
\appendix
\section<presentation>*{\appendixname}
\subsection<presentation>*{For Further Reading}

\begin{frame}[allowframebreaks]
  \frametitle<presentation>{For Further Reading}

  \begin{thebibliography}{10}

  \beamertemplatebookbibitems
  % Start with overview books.

  \bibitem{Author1990}
    A.~Author.
    \newblock {\em Handbook of Everything}.
    \newblock Some Press, 1990.


  \beamertemplatearticlebibitems
  % Followed by interesting articles. Keep the list short.

  \bibitem{Someone2000}
    S.~Someone.
    \newblock On this and that.
    \newblock {\em Journal of This and That}, 2(1):50--100,
    2000.
  \end{thebibliography}
\end{frame}

\end{document}
